\documentclass{homework}
\name{Artem Mavrin}
\course{Math 101}
\term{Fall 2016}
\hwnum{1}

\begin{document}
\maketitle

\begin{problem}[Euclid]
Prove that there are infinitely many prime numbers.
\end{problem}

\begin{solution}
Let $n$ be a natural number, and suppose we can find $n$ prime numbers $p_1, \ldots, p_n$.
Then $N = p_1 \cdots p_n + 1$ is coprime to each $p_i$, so it has a prime factor $q$ different from any of the $p_i$'s.
Then $p_1,\ldots,p_n,q$ is a list of $n+1$ prime numbers.
It follows that, given any finite set of prime numbers, we can always find a prime number not in that set.
Therefore, there are infinitely many prime numbers.
This contradicts the assumption that there are only finitely many primes.
\end{solution}

\begin{problem}
\begin{parts}
\part
\label{2.a}
Prove that $\sqrt{2}$ is irrational.
\part
\label{2.b}
Let $n$ be an integer greater than $1$.
Prove that the $n$th root of any prime number $p$ is irrational.
\end{parts}
\end{problem}

\begin{solution}
\ref{2.a}
Let $n = p = 2$ in part \ref{2.b}.

\ref{2.b}
The polynomial $f = x^n - p$ is irreducible over $\mathbb{Z}$ by Eisenstein's
criterion.
By Gauss's Lemma, $f$ is also irreducible over $\mathbb{Q}$.
In particular, $f$ has no roots in $\mathbb Q$, so the $n$th root of $p$ is
irreducible.
\end{solution}

\end{document}
