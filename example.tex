\documentclass[inconsolata]{homework}

\name{Leonhard Euler}
\course{Example}
\term{Fall 1737}
\hwnum{1}

\begin{document}

\begin{problem}
Prove that there are infinitely many prime numbers.
\end{problem}

\begin{solution}
Let $P$ be the set of prime numbers.
To prove that $P$ has infinitely many elements, it suffices to prove
that the series
\begin{equation}
\label{eq:sum of reciprocals of primes}
\sum_{p \in P} \frac{1}{p}
= \lim_{x \to \infty} \sum_{\substack{p \in P \\ p \leq x}} \frac{1}{p}
\end{equation}
diverges.
Suppose, for the sake of contradiction, that
\eqref{eq:sum of reciprocals of primes} converges.
Then there is a positive real number $M$ such that
\begin{equation}
\label{eq:tail of sum of reciprocals of primes}
\sum_{\substack{p \in P \\ p > M}} \frac{1}{p} < \frac{1}{2}.
\end{equation}
Let $p_1,\ldots,p_n$ be the distinct prime numbers contained in the interval
$[1,M]$, and let $Q = p_1 \cdots p_n$.
For every positive integer $N$, define
\begin{equation*}
S(N) = \sum_{n=1}^N \frac{1}{1 + n Q}
\end{equation*}
and let $P(N)$ be the finite set of primes $q$ which divide $1 + n Q$ for at
least one $n \in \{1,\ldots,N\}$.
For each positive integer $n$, the integers $Q$ and $1 + n Q$ are coprime, so
none of the prime numbers $p_1,\ldots,p_n$ divide $1 + n Q$.
It follows that
\begin{equation}
\label{eq:P(N) subset}
P(N) \subseteq P \cap (M, \infty)
\end{equation}
for every positive integer $N$.
If $m$ is another positive integer, then we define $A(N, m)$ to be the set of
all integers $n$ such that $1 \leq n \leq N$ and such that $1 + n Q$ has exactly
$m$ (not necessarily distinct) prime divisors.
Consider the sum
\begin{equation*}
S(N, m) = \sum_{n \in A(N, m)} \frac{1}{1 + n Q}.
\end{equation*}
Observe that $S(N, m) = 0$ for $m$ sufficiently large (since $A(N,m)$ is empty
for $m$ sufficiently large) and that
\begin{equation}
\label{eq:S(N) as sum}
S(N)
= \sum_{m=1}^\infty S(N, m).
\end{equation}
If $n \in A(N, m)$, then $1 + n Q =  (q_1\cdots q_m)^{-1}$ for some prime
numbers $q_1,\ldots,q_m \in P(N)$, so
\begin{equation*}
S(N, m)
\leq \sum_{q_1,\ldots,q_m \in P(N)} \frac{1}{q_1\cdots q_m}
= \left(\sum_{p \in P(N)} \frac{1}{p}\right)^{m}.
\end{equation*}
It then follows from \eqref{eq:tail of sum of reciprocals of primes} and
\eqref{eq:P(N) subset} that
\begin{equation*}
S(N, m)
\leq \left(\sum_{p \in P(N)} \frac{1}{p}\right)^{m}
\leq \left(\sum_{\substack{p \in P \\ p > M}} \frac{1}{p}\right)^{m}
< \frac{1}{2^m}
\end{equation*}
Now \eqref{eq:S(N) as sum} implies that
\begin{equation*}
S(N)
= \sum_{m=1}^\infty S(N, m)
\leq \sum_{m=1}^\infty \frac{1}{2^m} = 1.
\end{equation*}
Thus, the sequence $(S(N))_{N=1}^\infty$ is a monotone increasing bounded
sequence, so it converges to the limit
\begin{equation*}
\lim_{N \to \infty} S(N)
= \lim_{N \to \infty} \sum_{n=1}^N \frac{1}{1 + n Q}
= \sum_{n=1}^\infty \frac{1}{1 + n Q}.
\end{equation*}
However, the series above diverges (e.g., by the integral test), so we have
reached a contradiction.
It follows that the series \eqref{eq:sum of reciprocals of primes} diverges.
\end{solution}

\begin{problem}
\begin{parts}
\part
\label{2.a}
Prove that there is no rational number whose square is $2$.
\part
\label{2.b}
Let $n$ be an integer greater than $1$.
Prove that the $n$th root of any prime number $p$ is irrational.
\end{parts}
\end{problem}

\begin{solution}
\ref{2.a}
Suppose that there is a rational number whose square is $2$.
Then we can find coprime nonzero integers $a$ and $b$ such that $a^2 / b^2 = 2$.
Then $a^2 = 2 b^2$, so $a^2$ is even.
It follows that $a$ is even, so $a = 2 k$ for some integer $k$.
Then $2 b^2 = 4 k^2$, so $b^2 = 2 k^2$, which implies that $b^2$ is even.
But then $b$ itself is even, which is a contradiction since $a$ is even and $a$
and $b$ are coprime.
Thus, there is no rational number whose square is $2$.

\ref{2.b}
The polynomial $f = x^n - p$ is irreducible over $\mathbb{Z}$ by Eisenstein's
criterion.
By Gauss's Lemma, $f$ is also irreducible over $\mathbb{Q}$.
In particular, $f$ has no roots in $\mathbb Q$, so the $n$th root of $p$ is
irreducible.
\end{solution}

\begin{problem}
This is a multi-part problem with lowercase Roman numeral labels instead
of lowercase letters.
This is achieved by providing the option ``\texttt{r}'' to the \texttt{parts}
environment.
\begin{parts}[r]
\part
\label{3.i}
First part.
\part
\label{3.ii}
Second part.
\end{parts}
\end{problem}

\begin{solution}
There is nothing to show for \ref{3.i} and \ref{3.ii}.
\end{solution}

\begin{problem}
Let $S$ be a dense subset of $\mathbb R$.
Prove that if $f : \mathbb R \to \mathbb R$ is continuous and if $f(x) = 0$ for
all $x \in S$, then $f(x) = 0$ for all $x \in \mathbb R$.
\end{problem}

\begin{solution}
Let $x \in \mathbb R$ be given, and pick any $\varepsilon > 0$.
Since $f$ is continuous at $x$, there exists a $\delta > 0$ such that
$|f(x) - f(y)| < \varepsilon$ for all $y \in \mathbb R$ such that
$|x - y| < \delta$.
Since $S$ is dense in $\mathbb R$, there exists a number
$y \in S \cap (x - \delta, x + \delta)$.
Then $f(y) = 0$ and $|x - y| < \delta$.
Therefore, $|f(x)| = |f(x) - f(y)| < \varepsilon$.
Since $|f(x)| < \varepsilon$ for an arbitrary $\varepsilon > 0$, it follows that
$f(x) = 0$.
Since $x$ was arbitrary, this shows that $f(x) = 0$ for all $x \in \mathbb R$.
\end{solution}

\begin{problem}[Ahlfors \S4.2.3 \#2, p. 123]
Prove that a function which is analytic in the whole plane and satisfies an
inequality $|f(z)| < |z|^n$ for some $n$ and all sufficiently large $|z|$
reduces to a polynomial.
\end{problem}

\begin{solution}
Since $f$ is entire, we may write $f$ as a power series centered at $0$ which
converges for every complex number $z$:
\begin{equation*}
f(z) = \sum_{k=0}^\infty a_k z^k.
\end{equation*}
Let $R > 0$ be large enough such that $|f(z)| < |z|^n$ whenever $|z| \geq R$.
Let $\gamma$ be the positively oriented circle of radius $R$ centered at the
origin, parametrized as
\begin{equation*}
\gamma(t) = R e^{i t},
\qquad
t \in [0,2\pi].
\end{equation*}
For each $k \geq 0$ we have
\begin{align*}
a_k
&= \frac{f^{(k)}(0)}{n!}
= \frac{1}{2 \pi i} \int_\gamma \frac{f(z)}{z^{k+1}} \, dz
= \frac{1}{2 \pi i} \int_0^{2\pi} \frac{f(\gamma(t))}{\gamma(t)^{k+1}} \gamma^\prime(t) \, dt
\\&= \frac{1}{2 \pi i} \int_0^{2\pi} \frac{f(R e^{i t})}{R^{k+1} e^{i(t+1) t}} i R e^{i t} \, dt
= \frac{1}{2 \pi R^k} \int_0^{2\pi} \frac{f(R e^{i t})}{e^{i k t}} \, dt,
\end{align*}
and hence
\begin{align*}
|a_k|
&= \frac{1}{2 \pi R^k} \left|\int_0^{2\pi} \frac{f(R e^{i t})}{e^{i k t}} \, dt\right|
\leq \frac{1}{2 \pi R^k} \int_0^{2\pi} \frac{|f(R e^{i t})|}{|e^{i k t}|} \, dt
\\&= \frac{1}{2 \pi R^k} \int_0^{2\pi} |f(R e^{i t})| \, dt
\leq \frac{1}{2 \pi R^k} \int_0^{2\pi} R^n \, dt
= \frac{R^n}{R^k}
= \frac{1}{R^{k - n}}.
\end{align*}
If $k > n$, then letting $R \to \infty$ gives $|a_k| = 0$.
Thus, we have $a_k = 0$ for $k > n$, so that
\begin{equation*}
f(z) = a_0 + a_1 x + \cdots + a_n x^n.
\end{equation*}
That is, $f$ is a polynomial.
\end{solution}

\begin{problem}[Atiyah-Macdonald 1.1]
Let $x$ be a nilpotent element of a ring $A$.
Show that $1 + x$ is a unit of $A$.
Deduce that the sum of a nilpotent element and a unit is a unit.
\end{problem}

\begin{solution}
Let $y = -x$.
Then $y$ is also nilpotent, so choose a positive integer $n$ such that
$y^n = 0$.
We have
\begin{align*}
(1 - y) (1 + y + y^2 + \cdots + y^{n-1})
= 1 - y^n
= 1
\end{align*}
which shows that $1 + x = 1 - y$ is a unit.

Next, let $x\in A$ be nilpotent and $u\in A$ a unit.
Then $ux$ is nilpotent, so $1 + ux$ is a unit by the first part.
Therefore
\begin{equation*}
u + x = u^{-1}(1 + ux)
\end{equation*}
is a product of units, so it is a unit.
\end{solution}

\end{document}
